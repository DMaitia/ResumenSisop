\documentclass[a4paper]{article}

%% Language and font encodings
\usepackage[english]{babel}
\usepackage[utf8x]{inputenc}
\usepackage[T1]{fontenc}

%% Sets page size and margins
\usepackage[a4paper,top=3cm,bottom=2cm,left=3cm,right=3cm,marginparwidth=1.75cm]{geometry}

%% Useful packages
\usepackage{amsmath}
\usepackage{graphicx}
\usepackage{listings}
\usepackage[colorinlistoftodos]{todonotes}
\usepackage[colorlinks=true, allcolors=blue]{hyperref}

\author{You}

\begin{document}

\begin{figure}
 \centering
   \includegraphics[width=0.75\textwidth]{fig/fiuba.jpg}
\end{figure}

\begin{titlepage}
		\centering
		{\scshape\LARGE Universidad de Buenos Aires \par}
		\vspace{1cm}
		{\scshape\Large Materia\par}
		\vspace{0.5cm}
		{\scshape\Large Fecha de cursada\par}
		\vspace{1.5cm}
		{\huge\bfseries Tema del trabajo práctico \par}
		\vspace{1.5cm}
		{\huge\bfseries Informe Individual \par}
		\vspace{0.5cm}
		{\Large\itshape Darius Maitia - 95436 \par}
        \vspace{0.5cm}
        
    {\Large\itshape \par}
    \vspace{1cm}  
\end{titlepage}

\tableofcontents
\newpage


\section{Concurrencia}

\subsection{Dos razones para la concurrencia}

\begin{itemize}
\item Eficiencia: hay tareas que se hacen mejor de a varios
\item I/O : en un mismo proceso un hilo se ocupa del input/output y los otros hilos se ocupan de otras cosas.
\end{itemize}
\subsection{Concurrencia vs procesos}

\begin{itemize}
\item Los procesos se ejecutan de forma aislada
\item Los threads comparten recursos:
\subitem Comparten la memoria; entre threads no se necesita switchear page tables ni nada.
\subitem \textbf{No comparten stack! :} cada hilo tiene su propio stack. 
\end{itemize}

\subsection{Crear y joinear threads en C}
\begin{lstlisting}
#include <pthread.h>;

void * mythread(void * arg){
	printf("Imprimo gilada con una letra al final: %c", arg);
}

int main(){
	int i;
	pthread_t p1, p2;
	pthread_create(&p1, NULL, mythread, "A");
	pthread_create(&p2, NULL, mythread, "B");
	
	pthread_join(&p1, NULL);
	pthread_join(&p2, NULL);
	
	return 0;
}
\end{lstlisting}

Pthread\_create y pthread\_join deberían devolver 0 en caso de exito.

\subsection{Critical sections}
\textbf{Definición} : una critical section es una parte del código donde hay una variable compartida a la que pueden acceder múltiples threads.

Para encarar este problema hay que hacer lo siguiente:
\begin{enumerate}
\item Separar las incumbencias de los threads lo más que se pueda en el código.
\item En donde hayan variables de acceso compartido entre los threads, usar locks.
\end{enumerate}

\subsubsection{Locks}
Los locks permiten acceder a una variable de una critical section con total seguridad, sabiendo que otro hilo no podrá acceder a esa variable mientras yo esté usándola. 

La lógica es la siguiente:
\begin{lstlisting}
lock_t mutex;
...
lock(&mutex);
balance = balance + 1;
unlock(&mutex);
\end{lstlisting}

En lenguaje de programación C la lógica es la siguiente:
\begin{lstlisting}
pthread_mutex_t lock = PTHREAD_MUTEX_INITIALIZER;
Pthread_mutex_lock(&lock);
balance = balance + 1;
Pthread_mutex_unlock(&lock);
\end{lstlisting}

\newpage
\section{Conclusión}


\begin{thebibliography}{9}

\bibitem{ServoVsGecko}
  Michael Larabel,
  \textit{Mozilla's Servo Engine Is Crazy Fast Compared To Gecko},
  9 nov. 2014,\newline
  \url{https://www.phoronix.com/scan.php?page=news_item&px=MTgzNDA}

\end{thebibliography}


\end{document}